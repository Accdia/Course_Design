\section{总结}
本文主要围绕Tensorflow Lite进行简单介绍以及分析,将其所拥有的优点和缺点进行了对比,并且对其发展前景进行预测。
在此基础上,本文围绕Tensorflow Lite进行了简单的设计和实现,设计了一个可以在移动端本地对用户的物品进行识别分类
的Android应用软件。并且对其进行了简单的封装,使得其在移动端的使用更加方便。最后,本文对Tensorflow Lite的
使用进行了简单的测试,测试结果表明,Tensorflow Lite在移动端的使用效果良好,可以满足用户的需求。
\newpage
针对本次设计仍存在缺点,如:
\begin{itemize}
    \item 图像识别的准确率仍然需要提高。由于模型的训练图像集都是经过挑选的主体明显的图像,和实际识别的
    复杂场景有较大区别。所以模型要进一步提升实际图像识别状况下的准确率。
    \item 软件功能不成熟。虽然应用有实验性质,但是软件功能较少,设计比较粗糙,交互不够人性化。
    \item 程序设计结构可以优化。其直接影响程序运行速度。
\end{itemize}

通过本次课程设计学习了Tensorflow Lite的相关知识以及操作,对Tensorflow Lite的使用有了一定的了解,并且通过
动手设计了一个简单程序,从中收获良多。