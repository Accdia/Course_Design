\section{关键代码解释}
针对本项目的文件以及代码,我们在此对其进行了解释,以便读者能够更好地理解本项目的实现过程。
\begin{lstlisting}[style=code]
    package org.tensorflow.lite.examples.imageclassification;
\end{lstlisting}
这个包使用了预训练的模型,可以识别1000种图像类别。这个包还包含了一个Android测试类,
用于验证图像分类结果是否与预期一致。

在androidTest中写的是测试程序,用于验证模型的分类结果是否与预期一致。测试程序使用了JUnit框架,
并使用了Android测试库。

以下代码用于导入JUnit框架中的一些静态方法。这些方法可以用来测试程序。使用静态导入可以省略类名,可以方便地访
问类中的静态成员。
\begin{lstlisting}[style=code]
    import static org.junit.Assert.assertEquals;
    import static org.junit.Assert.assertNotNull;
\end{lstlisting}

\newpage
以下代码验证图像分类器进行的分类是否与预期的控制类别匹配。首先检查分类结果是否为空,然后比较控制数据和
分类数据中的类别数。最后,遍历每个类别并比较标签。
\begin{lstlisting}[style=code]
    assertNotNull(results.get(0));

    assertEquals(controlCategories.size(),
            results.get(0).getCategories().size());
    
    for (int i = 0; i < results.size(); i++) {
        assertEquals(
                controlCategories.get(i).getLabel(),
                results.get(0).getCategories().get(i).getLabel()
        );
    }
\end{lstlisting}

以下代码为可简化加载预训练的图像分类模型并使用它来对图像进行分类的过程,其创建一个实例并传递一个侦听分类结果的对象。
\begin{lstlisting}[style=code]
    ImageClassifierHelper helper = ImageClassifierHelper.create(
        InstrumentationRegistry.getInstrumentation().getContext(),
        new ImageClassifierHelper.ClassifierListener() {
            @Override
            public void onError(String error) {
                // no-op
            }

            @Override
            public void onResults(
                    List<Classifications> results,
                    long inferenceTime
            ) {
                // ...
            }
        });
\end{lstlisting}

以下代码使用了main文件夹中ImageClassifierHelper类,其封装了TensorFlow Lite图像分类任务的辅助类。其加载了
coffee.jpg的图片,并将其传递给classify方法进行分类。classify方法会调用onResults方法来处
理分类结果和推理时间,并与控制数据进行比较。如果分类结果和控制数据不一致,测试就会失败。
\begin{lstlisting}[style=code]
    helper.setThreshold(0.0f);
    helper.classify(loadImage("coffee.jpg"), 0);
\end{lstlisting}


此代码用于从 assets 文件夹加载图像并返回对象。该方法用于加载将由图像分类器分类的图像。
\newpage
\begin{lstlisting}[style=code]
    private Bitmap loadImage(String fileName) {
        AssetManager assetManager = InstrumentationRegistry
                .getInstrumentation()
                .getContext()
                .getAssets();
        try {
            InputStream inputStream = assetManager.open(fileName);
            return BitmapFactory.decodeStream(inputStream);
        } catch (IOException e) {
            e.printStackTrace();
        }
        return null;
    }
\end{lstlisting}

文件夹main/assets里包含的是模型文件和标签文件,这些文件是在训练模型时生成的。模型文件是一个二进制文件,
包含了模型的权重和结构信息。标签文件是一个文本文件,包含了模型可以识别的1000种图像类别。

针对CameraFragment类,用于设置相机及其用例,并允许用户修改分类分数阈值、一次可分类的最大对象数、
用于分类的线程数以及用于对象分类的底层硬件和模型等设置。

以下代码用于检查应用是否具有访问设备相机所需的权限。如果没有,它将导航到使用组件PermissionsFragment\hspace{0.5em}Navigation
\begin{lstlisting}[style=code]
    if (!PermissionsFragment.hasPermission(requireContext())) {
        Navigation.findNavController(requireActivity(), R.id.fragment_container)
                .navigate(
                        CameraFragmentDirections.actionCameraToPermissions()
                );
    }    
\end{lstlisting}

以下代码负责在销毁片段视图时执行清理操作。
\begin{lstlisting}[style=code]
    @Override
    public void onDestroyView() {
        super.onDestroyView();

        // Shut down our background executor
        cameraExecutor.shutdown();
        synchronized (task) {
            imageClassifierHelper.clearImageClassifier();
        }
    }
\end{lstlisting}

以下代码为分析用例捕获的每个相机帧调用该方法。它将 传递给 以对图像中的对象进行分类。
如果检测到对象并将其分类到阈值以上,则它们将显示在屏幕底部的回收器视图中。
\begin{lstlisting}[style=code]
    private void classifyImage(@NonNull ImageProxy image) {
        // Copy out RGB bits to the shared bitmap buffer
        bitmapBuffer.copyPixelsFromBuffer(image.getPlanes()[0].getBuffer());

        int imageRotation = image.getImageInfo().getRotationDegrees();
        image.close();
        synchronized (task) {
            // Pass Bitmap and rotation to the image classifier helper for
            // processing and classification
            imageClassifierHelper.classify(bitmapBuffer, imageRotation);
        }
    }
\end{lstlisting}

ClassificationResultAdapter类是用于显示分类结果列表的自定义 RecyclerView 适配器的实现。
适配器包含类别列表和适配器大小,并公开两个用于更新这些属性的公共方法。

该方法接收对象列表,并使用这些对象更新适配器的列表。它首先使用自定义实现按类别索引对给定列表
进行排序。然后,它使用updateResults、Category、categories、Comparator、ArrayList、adapterSize变量,
表示可以在回收器视图中显示的最大类别数。

然后,该方法循环遍历排序的类别,并将它们添加到adapterSize或大小sortedCategories列表,以较小者为准。最后,
它调用notifyDataSetChanged()以通知回收器视图数据集已更改,需要重新绘制。

这adapterSize具有给定大小的变量。此方法用于更新回收器视图中可以显示的最大类别数。
\begin{lstlisting}[style=code]
    @SuppressLint("NotifyDataSetChanged")
    public void updateResults(List<Category> categories) {
        List<Category> sortedCategories = new ArrayList<>(categories);
        Collections.sort(sortedCategories, new Comparator<Category>() {
            @Override
            public int compare(Category category1, Category category2) {
                return category1.getIndex() - category2.getIndex();
            }
        });
        this.categories = new ArrayList<>(Collections.nCopies(adapterSize, null));
        int min = Math.min(sortedCategories.size(), adapterSize);
        for (int i = 0; i < min; i++) {
            this.categories.set(i, sortedCategories.get(i));
        }
        notifyDataSetChanged();}
    public void updateAdapterSize(int size) {
        adapterSize = size;
    }

\end{lstlisting}

PermissionsFragment类用于请求应用访问设备相机所需的权限。并在必要时请求它

该方法检查是否已向应用授予权限。如果已授予权限,它将调用navigateToCamera。如果尚未授予权限,
它将使用requestPermissionLauncher向用户请求权限。
\begin{lstlisting}[style=code]
    @Override
    public void onStart() {
        super.onStart();
        if (ContextCompat.checkSelfPermission(requireContext(),
                Manifest.permission.CAMERA) == PackageManager.PERMISSION_GRANTED) {
            navigateToCamera();
        } else {
            requestPermissionLauncher.launch(Manifest.permission.CAMERA);
        }
    }
    
\end{lstlisting}

ImageClassifierHelper类有助于使用TensorFlow\hspace{0.5em}Lite完成图像分类任务。它包括用于
设置和运行图像分类器的方法,以及用于传回分类结果和错误消息的侦听器接口。在前面已有部分解释。

在setupImageClassifier方法中首先创建一个对象ImageClassifierOptions.Builder,为图像分类器设置各种选项,
例如分数阈值和最大结果数。
\begin{lstlisting}[style=code]
    @Override
    ImageClassifier.ImageClassifierOptions.Builder optionsBuilder =
    ImageClassifier.ImageClassifierOptions.builder()
            .setScoreThreshold(threshold)
            .setMaxResults(maxResults);
\end{lstlisting}

接下来,创建BaseOptions.Builder来为委托设置各种选项,例如要使用的线程数。
\begin{lstlisting}[style=code]
    BaseOptions.Builder baseOptionsBuilder =
    BaseOptions.builder().setNumThreads(numThreads);
\end{lstlisting}

根据currentDelegate的值,该方法选择要使用的委托(CPU、GPU 或 NNAPI),并在BaseOptions.Builder中设置
适当的选项。
\begin{lstlisting}[style=code]
    switch (currentDelegate) {
        case DELEGATE_CPU:break;
        case DELEGATE_GPU:
            if (new CompatibilityList().isDelegateSupportedOnThisDevice()) {
                baseOptionsBuilder.useGpu();
            } else {imageClassifierListener.onError("GPU is not supported on "+ "this device");}
            break;
        case DELEGATE_NNAPI:
            baseOptionsBuilder.useNnapi();
    }
\end{lstlisting}

根据currentModel的值,该方法选择要使用的模型(MobileNetV1 或 EfficientNetLite 的三个版本之一),
并设置适当的模型文件名。
\begin{lstlisting}[style=code]
    String modelName;
    switch (currentModel) {
        case MODEL_MOBILENETV1:
            modelName = "mobilenetv1.tflite";
            break;
        case MODEL_EFFICIENTNETV0:
            modelName = "efficientnet-lite0.tflite";
            break;
        case MODEL_EFFICIENTNETV1:
            modelName = "efficientnet-lite1.tflite";
            break;
        case MODEL_EFFICIENTNETV2:
            modelName = "efficientnet-lite2.tflite";
            break;
        default:
            modelName = "mobilenetv1.tflite";
    }
\end{lstlisting}

最后,ImageClassifier尝试从所选模型和选项创建对象。如果成功,则imageClassifier的字段设置为新对象。
如果不成功,imageClassifierListener将向控制台报告错误消息,并将错误日志打印到控制台。
\begin{lstlisting}[style=code]
    try {
        imageClassifier =
                ImageClassifier.createFromFileAndOptions(
                        context,
                        modelName,
                        optionsBuilder.build());
    } catch (IOException e) {
        imageClassifierListener.onError("Image classifier failed to "
                + "initialize. See error logs for details");
        Log.e(TAG, "TFLite failed to load model with error: "
                + e.getMessage());
    }
\end{lstlisting}



