%%
% NCHU Bachelor Proposal Report Template
%
% 南昌航空大学毕业设计开题报告(目标、主要特色及工作进度)—— 使用 XeLaTeX 编译
%
% Copyright 2023 Arnold Chow
%
% The Current Maintainer of this work is Arnold Chow.
%
% Compile with: xelatex -> biber -> xelatex -> xelatex

\section{设计方案}
针对所需内容,本项目采用TensorFlow\hspace{0.5em}Lite作为图像识别的框架,
使用Android Studio作为开发环境,以Java语言开发,使用Android手机作为测试设备。本项目的设计方案如下:
\begin{enumerate}
    \item 收集和准备数据集:下一步是收集和准备数据集。这涉及收集与用例相关的大型图像数据集并相应地标记它们。数据集应该是多样化的,并代表应用程序将遇到的实际方案。
    \item 训练机器学习模型:收集和准备数据集后,下一步是使用 TensorFlow 训练机器学习模型。这涉及创建一个深度学习模型架构,并使用TensorFlow的API在数据集上对其进行训练。
    \item 将模型转换为 TensorFlow Lite:模型经过训练后,需要将其转换为 TensorFlow Lite 格式才能部署在移动设备上。这可以使用TensorFlow的转换工具完成。
    \item 将模型集成到Android应用程序中:下一步是将TensorFlow Lite模型集成到Android应用程序中。这涉及将模型添加到应用程序的代码库,并为用户创建与模型交互的接口。
\end{enumerate}
本次设计的关键点在于创建深度学习模型架构、Tensorflow的API调用以及数据集的收集以及调用。而数据集则来源于网络上的公开数据集,如Github等。
