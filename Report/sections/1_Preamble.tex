%%
% NCHU Bachelor Proposal Report Template
%
% 南昌航空大学毕业设计开题报告(选题的依据与意义)—— 使用 XeLaTeX 编译
%
% Copyright 2023 Arnold Chow
%
% The Current Maintainer of this work is Arnold Chow.
%
% Compile with: xelatex -> biber -> xelatex -> xelatex

\section{序言}

移动应用的开发在近年来发展迅速,几乎所有用户常用的功能都已经有相应的软件能够满足需求。
但与此同时,用户对应用的要求也相应提高,而不仅仅满足于功能实现。移动应用除了在功能和性
能上要有所突破外,在用户隐私保护和安全性上的要求也相应提高。如何以更加高效合理的方式开发
,高性能且安全的应用以满足用户的需求成为了每个移动开发者的首要任务。与此同时,基于深度学
习的人工智能在近年来快速发展,让许多技术领域有了重大突破。如何和人工智能进行有效结合,让
移动开发有了新的尝试途径。传统的图像分类应用都需要将图像通过网络上传到云端服务器,再进行
图像分类并将分类标签重新下发到移动端,移动端再将分类结果根据标签进行展示。这样的做法不仅依
赖网络,低效并且存在安全隐患。为了克服上述问题,本程序设计和实现一个可以在移动端本地对用户的
物品进行识别分类的Android应用软件。该软件在移动端通过Tensorflow Lite神经网络框架运行图像分
类神经网络对用户的物品进行实时识别。

\textbf{关键词}\hspace{1em}Android,Tensor flow lite,图像识别